% ======================================================================
% brlex-doc.tex
% Copyright (c) Youssef Cherem <ycherem(at)gmail.com>, 2016
%
% This file is part of the br-lex LaTeX2e class.

% This work may be distributed and/or modified under the conditions of
% the LaTeX Project Public License, version 1.3c of the license.
% The latest version of this license is in
%   http://www.latex-project.org/lppl.txt
% and version 1.3c or later is part of all distributions of LaTeX
% version 2005/12/01 and of this work.
%
% This work has the LPPL maintenance status "author-maintained".
% ======================================================================

\documentclass{ltxdoc}
\usepackage[brazil]{babel}
\usepackage{erewhon}
\RequirePackage[sharp]{easylist}
\usepackage[normalem]{ulem}
\newcommand{\cortado}[1]{\sout{#1}}
\usepackage[dvipsnames]{xcolor}

\frenchspacing 

\ListProperties(Numbers1=R,Numbers2=l,Numbers3=a,
FinalMark1={~---},FinalMark2={)},Hide2=1,Hide3=2,Margin1=4em,Margin2=6em,Margin3=7.5em,Align=move) 

\title{\texttt{br-lex}: Uma classe para redação\\ de normas e leis brasileiras}
\author{Youssef Cherem}

\newcommand{\nota}[1]{\marginpar{\hfill \cmd{#1}}}

\begin{document}

\maketitle

\begin{abstract}
Classe para redação de normas e leis de acordo com a Lei Complementar Nº 95, de 26 de fevereiro de 1998. O objetivo é uma implementação tipográfica facilitada para o usuário.
\end{abstract}

\section{Utilização}

Para usar a classe \texttt{br-lex}, basta invocar as palavras mágicas: 

\begin{verbatim}
\documentclass{br-lex}
\end{verbatim}

As opções de língua portuguesa já vêm carregadas automaticamente, utilizando o pacote \texttt{babel}, se a compilação for com \texttt{pdflatex}, e o pacote \texttt{polyglossia}, se optar por \texttt{xelatex}. 

Esta classe foi elaborada com a classe \texttt{mwbk}. Além das opções das classes \texttt{mwcls}, há as seguintes opções: 

\begin{itemize}

\item 
\texttt{paragrafonormal}, para parágrafos indentados em sem espaço entre si; 

\item
\texttt{paragrafoespaco}, para parágrafos sem indentação e com espaço entre si, e 

\item 
\texttt{capitulo}, para começar os capítulos na mesma página.

\end{itemize}

As páginas têm cabeçalho automático de acordo com a classe \texttt{scrbook}. Se desejar, você pode mudar o estilo da página com o pacote \texttt{scrlayer-scrpage} (antigo \texttt{scrpage2}).



\begin{verbatim}\titulo{}
\end{verbatim}
\nota{\titulo\marg{Título}} 
Título.

Capítulos e seções são inseridos da maneira usual em \LaTeX: \verb|\chapter|\marg{Capítulo} e \verb|\section|\marg{Seção}.   


\begin{verbatim}\descricao{}
\end{verbatim}
\nota{\descricao\marg{Descrição}} 
Descrição da lei.


\begin{verbatim}\cortado{}
\end{verbatim}
\nota{\cortado\marg} 
Para inserir partes ``cortadas" (\cortado{revogadas}).


\begin{verbatim}
\artigo Escreva aqui o artigo.
\end{verbatim}
 
\cmd{\artigo} (sem \verb|{}|)\nota{\artigo} Insere um artigo, cuja numeração não depende do capítulo.



\begin{verbatim}
\begin{paragrafos}
\paragrafo Escreva aqui o parágrafo.
\paragrafo Outro parágrafo
...
\end{paragrafos}
\end{verbatim}

Ambiente \texttt{paragrafos},\nota{\paragrafo} cuja numeração recomeça a cada artigo. Dentro desse ambiente, você pode inserir um \verb|\paragrafo| (sem \verb|{}|). 

Para divisões inferiores a parágrafo, é utilizado o pacote \texttt{easylist}:

\begin{verbatim}
	\begin{easylist}
	# Incisos (romanos) com #
	
	## Alíneas (minúsculas) com ##
	
	### Itens com (arábicos) com ###
	\end{easylist}
\end{verbatim}
	
	

\begin{easylist}
	# Incisos (romanos) com \verb|#|
	
	## Alíneas (minúsculas) com \verb|##|
	
	### Itens com (arábicos) com \verb|###|
	
	###
	
	## \verb|##| Volta para alíneas.
	
	# \verb|#| Volta para incisos.
	\end{easylist}


\end{document}